\documentclass[10pt]{article}
\usepackage{makeidx}
\usepackage{multirow}
\usepackage{multicol}
\usepackage[dvipsnames,svgnames,table]{xcolor}
\usepackage{graphicx}
\usepackage{epstopdf}
\usepackage{ulem}
\usepackage{hyperref}
\usepackage{amsmath}
\usepackage{amssymb}
\author{星宇 朱}
\title{}
\usepackage[paperwidth=595pt,paperheight=841pt,top=72pt,right=90pt,bottom=72pt,left=90pt]{geometry}

\makeatletter
	\newenvironment{indentation}[3]%
	{\par\setlength{\parindent}{#3}
	\setlength{\leftmargin}{#1}       \setlength{\rightmargin}{#2}%
	\advance\linewidth -\leftmargin       \advance\linewidth -\rightmargin%
	\advance\@totalleftmargin\leftmargin  \@setpar{{\@@par}}%
	\parshape 1\@totalleftmargin \linewidth\ignorespaces}{\par}%
\makeatother 

% new LaTeX commands


\begin{document}


Given the equation $\frac{d^2s}{{dt}^2}=-ks$, now we want to find a
parametrization of this cycloid.

The second law gives us:

Combining these two equations

s=g/K sin$\theta{}$

Differentiate the equation

ds = g/K cos$\theta{}$ d$\theta{}$ (1)

On the other hand, ds=.$\sqrt{{(\frac{dx}{dt})}^2+{(\frac{dy}{dt})}^2}d\theta{}$
And Since the normal component of the speed is 0:

C

\begin{equation}
\frac{dx}{dt}sin\theta{}=\frac{dy}{dt}cos\theta{}
%eq1
\end{equation}

ombining the two equation, we get ds = dx / cos $\theta{}$,



\[
\frac{ds}{dt}=\frac{dx}{cos\theta{}dt}
\]

substitute (1) into it



\[
\frac{dx}{dt}=\frac{g}{2k}(1+cos2\theta{})\frac{d\theta{}}{dt}
\]

From equation(2) we get



\[
\frac{dy}{dt}=\frac{g}{2k}sin2\theta{}\frac{d\theta{}}{dt}
\]

Multiply both sides with dt and integrate both sides, we get


\[
x=\intdx=\int\frac{g}{2k}(1+cos2\theta{})d\theta{}=\frac{g}{4k}\left(2\theta{}+sin2\theta{}\right)+C1
\]




\[
y=\intdy=\int\frac{g}{2k}sin2\theta{}d\theta{}=-\frac{g}{4k}cos2\theta{}+C2
\]

Applying the initial conditions x = 0 and y = -g/2k , both when $\theta{}$ = 0, 
we finally get


\[
x=\frac{g}{4k}\left(2\theta{}+sin2\theta{}\right)
\]



\[
y=-\frac{g}{4k}(cos2\theta{}+1)
\]


Now we can show such curve is a cycloid



\[
x^{'}=\frac{g}{2k}(1+cos2\theta{})
\]



\begin{equation}
y^{'}=\frac{g}{2k}sin2\theta{}{x'}^2+{y'}^2=\frac{g^2}{2k^2}(1+cos2\theta{})
%eq2
\end{equation}

(2)

Since the conservation of mechanical energy gives us ,$\frac{1}{2}mv^2=mgy$

,

\[
v=\sqrt{2gy}
\]

plugging into
$v=\frac{ds}{dt}$

\[
dt=\frac{ds}{\sqrt{2gy}}
\]




\[
=\frac{\sqrt{{dx}^2+{dy}^2}}{\sqrt{2gy}}
\]

From equation (1) and (2), we can get


\[
dt=\frac{\sqrt{\frac{g^2}{2k^2}(1+cos2\theta{})}}{\sqrt{\frac{g^2}{2k}(1+cos2\theta{})}}d\theta{}
\]




\[
=\sqrt{\frac{1}{k}}d\theta{}
\]

Integrate both sides



\[
T=\int_0^{\pi{}}\sqrt{\frac{1}{k}}d\theta{}=\sqrt{\frac{1}{k}}\pi{}
\]

Now considering an intermediate point

${\theta{}}_0$Since the conservation of mechanical energy gives us
,$\frac{1}{2}mv^2=mgy-mgy_0$

,

\[
v=\sqrt{2g(y-y_0)}
\]

plugging into
$v=\frac{ds}{dt}$

\[
dt=\frac{ds}{\sqrt{2g(y-y_0)}}
\]



\[
=\frac{\sqrt{{dx}^2+{dy}^2}}{\sqrt{2g(y-y_0)}}
\]




\[
=\frac{\sqrt{\frac{g^2}{2k^2}(1+cos2\theta{})}}{\sqrt{-\frac{g^2}{2k}(1+cos2\theta{}-1-cos2{\theta{}}_0)}}d\theta{}
\]

Then integrate both sides, after computing it with computer software



\[
T=\sqrt{\frac{1}{k}}\pi{}
\]

That shows T is the same for any point of the curve. Therefore, the curve is a
cycloid.

Reference

Weisstein, Eric W. "Tautochrone Problem." From MathWorld--A Wolfram Web
Resource. http://mathworld.wolfram.com/TautochroneProblem.html


\end{document}